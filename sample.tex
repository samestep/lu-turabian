\documentclass[raggedright]{turabian-researchpaper}

\usepackage[mincrossrefs=1]{biblatex-chicago}
\usepackage{fmtcount}
\usepackage{fnpct}
\usepackage{hyperref}
\usepackage{indentfirst}
\usepackage{mathptmx}
\usepackage[super]{nth}
\usepackage{setspace}
\usepackage{soul}
\usepackage{tocstyle}
\usepackage{xcolor}
\usepackage{xfrac}

\NewBibliographyString{cbyreviser}
\DefineBibliographyStrings{english}{cbyreviser = {rev.}}

\AdaptNoteOpt\autocite\multfootcite

\definecolor{lightblue}{rgb}{0,1,1}
\sethlcolor{lightblue}

\newtocstyle[standard][leaders]{standardwithdot}{}
\usetocstyle{standardwithdot}

\newcommand*{\bluehref}[2]{\href{#1}{\color{blue}\ul{#2}}}
\newcommand*{\subsubsubsection}[1]{\subsubsection{\normalfont #1}}
\newcommand*{\subsubsubsubsection}[1]{\textbf{#1}}

\addbibresource{sample.bib}

\title{\normalfont A Sample Paper for the Purpose of Correct Formatting Using
  \\\ \\ \textbf{Notes-Bibliography} Style for Students in All
  \textbf{Non-Divinity} Degree Plans}
\author{Claudia Q. Sample}
\course{HIUS 221: Survey of American History}
\date{April 2, 2018}

\begin{document}
\pagenumbering{roman}
\maketitle

\setcounter{page}{2}
\pagestyle{plain}
\setcounter{tocdepth}{2}
\tableofcontents
\clearpage
\pagenumbering{arabic}
\pagestyle{headings}

\section{Introduction}

``Turabian'' style is an abbreviated version of the more-comprehensive
``Chicago'' style. Turabian is named for Kate L. Turabian (2013), the author of
\textit{A Manual for Writers of Research Papers, Thesis, and Dissertations:
Chicago Style for Students \& Researchers}, which is currently in its 8th
printed edition.\autocite{Turabian} This sample paper will strive to provide
students with all the foundational elements of a Turabian paper using the
\textbf{\ul{Notes-Bibliography format}} for all students majoring in History and
other degrees requiring Turabian format that do not fall under LU's Rawlings
School of Divinity. LU's Rawlings School of Divinity has made some nominal
changes to standard Turabian formatting that do not apply to non-Divinity
students. Students in those degree plans should NOT use the format included in
the sample paper, but rather should follow the format set forth in the School of
Divinity's unique sample papers instead.\footnote{LU's Rawlings School of
Divinity (SOD) published its own
\bluehref{http://www.liberty.edu/divinity/index.cfm?PID=28160}{Writing Guide},
which includes a link to the SOD's sample Turabian paper (also in
notes-bibliography style). Students in degrees under the SOD should follow those
guidelines, rather than the ones presented in this sample paper.}

This paper will focus primarily on the stylistic elements discussed in Chapters
16 and 17 of the Turabian
manual\autocite[\protect\label{fn:three}][144-215]{Turabian}---\hl{\emph{as
revised by the History Department}}.\footnote{For example, footnote numbers in
standard Turabian are not superscripted and are followed by a period (pages 156
and 390 of your Turabian manual), but \textbf{the History Department requires
superscripted numbers with no periods}, as depicted throughout this paper.}
Students will need to incorporate proper grammatical elements to the papers as
well, but those will not be addressed in detail herein.

Many students' papers will require an introductory section that summarizes or
previews the argument of the whole paper, though this is not universally
required for all papers.\autocite[\protect\label{fn:five}][390]{Turabian} It
should be set apart as a separate First-Level Subheading (addressed below).
Leave \ul{two blank spaces} beneath the word Introduction and the text that
follows, as shown above.\autocite[\protect\label{fn:six}][390]{Turabian}
Turabian suggests that ``most introductions run about 10 percent of the
whole.''\autocite[\protect\label{fn:seven}][104]{Turabian} She also suggests
that conclusions are typically shorter than introductions.

\section{Ibid.}

The abbreviation \textit{ibid.} is used to refer to ``the same'' source cited
\ul{immediately before} \emph{on the same
page}\autocite[\protect\label{fn:eight}][161]{Turabian}---in this case, footnote
\#\ref{fn:three}. The term \textit{ibid.} itself is a Latin abbreviation (which
is why it is italicized in the text of a sentence), so DO include the period.
Capitalize it when it begins the footnote, since it depicts the beginning of a
sentence, but do \textbf{not} italicize the term in
notes.\autocite[\protect\label{fn:nine}][161]{Turabian} If the page numbers for
that footnote and the one preceding it differ, use Ibid. followed by a comma and
the correct page number(s), as shown in footnotes \#\ref{fn:seven} and
\#\ref{fn:eight} below. If the page number is the same for both the current
footnote and the one that precedes it, simply use the word Ibid. for that second
footnote, as shown in footnotes \#\ref{fn:seven} and \#\ref{fn:nine} below.

Each new page of a student's paper restarts the requirements \dots{} so the
first footnoted citation to a source on each page would include the author's
name and a shortened title (if previously cited), then students can resume usinb
ibid. for subsequent citations on that page, as shown in footnotes
\#\ref{fn:five} and \#\ref{fn:six} above. Standard Turabian format allows two
forms of shortened notes,\autocite[\protect\label{fn:ten}][158-160]{Turabian}
but the History Department requires the author-title version that includes both
the author's name and a shortened version of the source's title. Footnote
\#\ref{fn:ten} below (and the first footnote on each new page referring to the
Turabian manual) depicts a shortened note---where the author's name is given,
along with a few words of the title. Always include the page number, whether you
use a full footnote or a shortened note.

\section{Basic Formatting}

\subsection{Overview}

Turabian generally offers writers great flexibility in the choices they make
regarding many stylistic elements.\footnote{For example, \hl{Turabian does not
specify a font size or style, although the OWC generally recommends Times New
Romans, 12-point font}. Many other elements are also left by Turabian up to
individual writers; \hl{the OWC has incorporated its own educated judgment for
those in this sample paper, but professors (and students) have freedom to stray
from those, as permitted in the Turabian manual}.} However, LU's schools,
departments, and professors have adopted specific requirements. A professor's
mandates trump anything in this sample paper; students should always adopt their
specific paper's unique expectations when those differ from the proposed
formatting in this sample paper. When in doubt, it may prove beneficial to ask
your professor for clarification.

\subsection{Title Page}

The Turabian manual provides two different examples and details for the optional
title page formats.\autocite[377-76]{Turabian} \hl{LU's History Department
requires a dissertation-style title page}, but students in non-History and
non-Divinity degree programs should check with their individual professors to
determine the requirements of each. Though the two samples in the Turabian
manual\autocite[376]{Turabian} both depict bolded titles on those title pages,
section A.2.1.2\autocite[376]{Turabian} is silent on such. Section A.1.5,
however, specifies that titles should be bolded (though that section appears to
be referring specifically to subheadings).\autocite[374-75]{Turabian} As with
all other issues where the Turabian manual is unclear, students should be sure
to clarify with their professor if they have any questions about the professor's
expectations and requirements.

Other formatting elements required include:
\begin{itemize}
\item One-inch margins on all four sides of the paper.
\item Times New Roman size 12-pt. font except the content in the footnotes
  themselves (not the superscripted numerals in the body of the paper), which
  should be Times New Roman 10-pt. font.
\item Double-spacing throughout the body of the paper, except in the footnotes,
  block quotes, table titles, and figure captions. Lists in appendices should be
  single-spaced, too.\autocite[373]{Turabian}
\item Quotations should be blocked if the citation is five or more lines.
  Blocked quotations are single-spaced, and every line is indented one
  half-inch, with an extra return line before and after the excerpt. No
  quotation marks are used when using blocked quotations. Turabian requires
  blocked quotes to be introduced in the writer's own
  words.\autocite[349]{Turabian}
\end{itemize}

\subsection{Page Numbering}

The title page should not include any page number, although it is considered the
first page of any paper. The front matter (anything between the title page and
the first page of the body of the paper) should be numbered with lowercase Roman
numerals centered in the footer, beginning with ii, to correspond with the fact
that it begins on page two. The paper body, bibliography, and appendices display
Arabic numerals (i.e., 1, 2, 3) placed flush-right in the
header,\autocite[374]{Turabian} beginning with page 1 on the first page of the
body of the paper.

LU has prepared a
\bluehref{http://www.liberty.edu/media/1171/CMO_-_Pagination_Tutorial_for_Turabian_-_edited_by_Sam_-_02022017.pdf}{tutorial
on pagination for Turabian papers}, with screenshots in MS Word to help students
learn how to insert page numbers, setp-by-step. Voila \dots{} pagination
mastered!

\subsection{Table of Contents}

Although this page/section is commonly referred to as the ``table of contents,''
only the word ``Contents'' should appear at the top, centered, without the
quotation marks. Students may not need a table of contents, but one was included
in this sample paper as a visual, and because it is lengthy enough to include
subheadings. Note that ``\textbf{all} papers \textbf{divided into chapters}
\ul{require} a table of contents'' (emphasis supplied.\autocite[380]{Turabian}
If a student's paper is not divided into chapters or subheadings (usually short
papers), it likely will not be required to include a table of contents---please
clarify with the professor.

The table of contents can span more than one page when necessary, as it does in
this sample paper. Double-space between each item but single-space the
individual items themselves. Add an extra line between each of the major
sections (including the front and back matter). It is important to note that a
table of contents does not list the pages that precede it, only those pages that
follow it. Be sure that the first letter of each word is capitalized (other than
articles and prepositions within the phrase).

``Leaders''---the dots between the words on the left margin and their
corresponding page numbers at the right margin---are acceptable, at the
student's (or professor's) choice. Only include the first page each element
begins on; not the full page-span.

Number all pages of this element with Roman numerals. If the table of contents
is more than one page, do not repeat the title. Leave two blank lines between
the title and the first listed item. Single-space individual items listed but
add a blank line after each item. Between the lists for the front and back
matter and the chapters, or between parts or volumes (if any), leave two blank
lines.

\subsection{Thesis Statements}

Section 2.1.13 of the Turabian manual discusses the placement and labeling of an
abstract or thesis statement. Specifically, it acknowledges that ``most
departments or universities have specific models \dots{} that you should follow
exactly for content, word count, format, placement, and
pagination.''\autocite[389]{Turabian} Since this can vary from one professor to
another, this paper will not include a separate sample thesis statement page. Do
not confuse a purpose statement with a thesis statement, however. A purpose
statement states the reason why the paper is written. For all practical
purposes, the purpose statement introduces the thesis statement. An example of a
purpose statement is, ``The purpose of this paper is to demonstrate that when
one recognizes God's freedom he/she can find biblical inerrancy is defensible.''
An example of a thesis statement is, ``Biblical inerrancy is defensible in the
context of divine freedom.'' Remember, it is the purpose and thesis statements
that determine the form and content of an outline.

The distinction between a purpose statement and a thesis statement is important.
The purpose of this sample paper is to provide a template for the correct
formatting of a research paper. The thesis is, ``Students who use this paper as
a sample or template are more likely to format their papers correctly in the
future.''

\subsection{Line Spacing}

Section A.1.3 of the Turabian manual addresses line
spacing.\autocite[373]{Turabian} As mentioned above, all text in papers should
be double-spaced except for block quotes, table titles, figure captions, and
lists in appendices. The table of contents, footnotes, and bibliography entries
should be single-spaced internally, but double-spaced between each
entry.\autocite[373]{Turabian} An extra double-spaced line should precede each
subheading. There should only be \ul{one space} after periods and other
punctuation at the end of each sentence, before beginning a new
sentence.\autocite[373]{Turabian}

\subsection{Capitalization}

Turabian style has two forms of capitalization for titles: headline-style and
sentence-style.\autocite[312]{Turabian} LU courses typically use headline style.
Be sure to check with your professor.

\subsection{Chapters versus Subheadings}

Turabian allows each writer to determine whether to use subheadings or chapters
to divide his or her paper into sections. There are separate rules for both.
Chapters are usually reserved for thesis projects and dissertations.

Turabian allows great flexibility and individuality in how one formats the
various subheading levels, when used. When your professor does not specify a
preferred format for heading levels, students are invited to adopt the formats
shown herein, for the sake of consistency and uniformity. Your professor's
mandates, however, \textbf{always} trump any other source's recommendations.
LU's Online Writing Center strongly urges students \ul{not} to search the
Internet for outside sources regarding Turabian format, since those are all
individuals' interpretations, and there is tremendous disparity from one source
to another. For example, some resources recommend using ALL-CAPITAL letters, but
Turabian expressly discourages such as having ``the undesirable effect of
obscuring the capitalization of individual words in the
title.''\autocite[375]{Turabian}

The title of a heading cannot be left ``orphaned'' at the bottom of a page,
without its supporting text.\autocite[393]{Turabian} If there is not enough room
on the previous page for both the heading title and at least the first two lines
of the paragraph, you must begin a new page. You can have two headings in a row
as shown on page 14 below separated by one blank line, however.

The formats used and recommended in this sample paper align with
\bluehref{http://www.liberty.edu/academics/graduate/writing/index.cfm?PID=34282}{LU's
Turabian Headings document}:
\begin{enumerate}
\item First-level headings should be centered, bolded, and use headline-style
  capitalization.
\item Second-level headings should be centered, not bolded, and use
  headline-style capitalization.
\item Third-level headings should be left-justified, bolded, and use
  headline-style capitalization.
\item Fourth-level subheadings---though rare in Turabian style---should be
  left-justified, not bolded, with only the first letter of the first word
  capitalized.
\item Fifth-level subheadings are extremely rare; they should be indented
  \sfrac{1}{2}'' from the left margin, not bolded, italicized, in sentence case
  (including a period), followed by one space, with the text following on the
  same line.
\end{enumerate}

Except for fifth-levels, all text would begin on the line beneath the heading.
Note that there must be at least two of any subheading used under a larger
heading. Turabian also does not allow orphaned headings, where the heading
appears at the bottom of the page, isolated from its content on the next
page.\autocite[393]{Turabian} These subheading levels are used throughout this
sample paper, but below is a visualization of each. Note that sectional headings
should have two blank single-line spaces above the heading and one blank
single-line space below the heading.

\section*{Contemporary Art}

\subsection{What Are the Major Styles?}

\subsubsection{Abstract Expressionism}

\subsubsubsection{Major painters and practitioners}

\subsubsubsubsection{Pollack as the leader}. This one is unique in that the text
begins on the same line.

\subsection{``Voice''}

As a general rule, use active voice and avoid first or second person pronouns
unless permitted by the assignment instructions. This paper uses second person
(you, your) since it is instructional in nature. In historical writing, use
simple past tense verbs, but when referring to an author's written work, use
present tense.

\section{Organizing a Paper Using an Outline}

When writing a paper, organize your outline first so that you are able to plan
how you will make your argument and then give your reasoning and evidence to
support your thesis statement. Your first paragraph of each section should
explain how this will fit into your reasoning, and then each section will end
with a summary of how the evidence has shown your reasoning to be correct. Also,
transitions are very helpful at the end of each major section so that the reader
anticipates how the next section is connected to the logical progression of the
reasoning you use to support your thesis.

Most LU research papers will be no longer than twenty pages and generally will
not have long and detailed outlines or subheadings beyond the third level.
Details that would be appropriate for the fourth and fifth heading level tend to
distract the reader's attention from the overall thesis within a short essay
(typically fewer than 20 pages). Even if a fourth level is unavoidable, a fifth
level is discouraged.

\section{Citations}

Though Turabian allows two forms of citing your sources in the body of a paper,
this sample paper focuses exclusively on the notes-bibliography style. Chapter
17 of your Turabian manual focuses on these elements. The ``N'' denotes
(foot)notes, and the ``B'' denotes bibliography entries. Be sure to use the
correct format for each since there are some variances between them for each
resource. Notably, the first/only author's name is inverted (i.e., last name
first) in bibliography entries, but not in notes.

When formatting a footnote, indent the first line of each footnote the same
amount as the first line of the paragraphs within your paper (generally 1/2'').
The indentation should be before the superscripted footnote number. Insert one
space after the superscript number before the first word of the
footnote.\footnote{\hl{This rule is specific to the History Department}. In
standard Turabian, the footnote number is not superscripted and it is followed
by a period.} The footnotes should be single-spaced, and there should be a
single blank space between (or 10-pt. line space after) each footnote.

There is one notable exclusion to notes formatting that most LU students will
encounter. The version of translation of the Bible being used must be identified
in the text with a parenthetical reference (e.g. 1 Cor 1:13, ESV). If you choose
to use the same Bible translation (such as the \textit{English Standard
Version}) throughout the paper you should add a footnote in your first usage
stating, ``Unless otherwise noted, all biblical passages referenced are in the
\textit{English Standard Version}.'' This means that you will not need to
reference the version of the Bible in subsequent citations unless you change the
version. For example, if the student identified the \textit{English Standard
Version} as the primary version but chose to use the \textit{New International
Version} (NIV) when quoting a particular verse such as John 14:6, the
parenthetical citation following the passage would be (John 14:6, NIV). Whatever
translation, the full name should be italicized since a translation is a book
title (do not italicize the abbreviation, as depicted in this paragraph).

An example of each of the major types of footnoted resources is included herein
for sample purposes. Note that writers would only include footnotes and
bibliography entries for resources whose content were actually used in
supporting the author's position in a paper. These samples that follow are for
illustration purposes only, and each source footnoted herein is also included in
the bibliography section.
\begin{itemize}
\item Book by one author.\autocite[65]{Doniger}
\item Book by two authors.\autocite[104--7]{Cowlishaw}
\item Book by three authors.\autocite[11-12]{Owenby}
\item Book by four or more authors.\autocite[262]{Laumann}
\item Editor, translator, or compiler \ul{instead} of
  author.\autocite[91--92]{Lattimore}
\item Editor, translator, or compiler \ul{in addition to}
  author.\autocite[22]{Bonnefoy}
\item Chapter or other part of a book.\autocite[101--2]{Wiese}
\item Preface, foreword, introduction, or similar part of a
  book.\autocite[xx--xxi]{Rieger}
\item Book published electronically.\autocite{Kurland}\footnote{Note: if the
  book is a PDF of a hard-copy publication, you do not need the web address,
  etc. You should cite it as if you are holding it in your hand. However, if the
  on-line publication is in digitized form and does not have page numbers
  reflecting the actual pages in the book, then follow the example directly
  below. \emph{If a book is available in more than one format, cite the version
  you consulted. For books consulted online, include an access date and a URL.
  If you consulted the book in a library or commercial database, you may give
  the name of the database instead of a URL. If no fixed page numbers are
  available, you can include a section title or a chapter or other number.}}
\item Article in a print journal.\autocite[639]{Smith}\footnote{Note: if the
  journal has more than one issue per year---i.e., (April 1998)---put the month
  before the date.}
\item Article in an online journal.\autocite{Hlatky}\footnote{For a journal
  article consulted online, include an access date and a URL. For articles that
  include a DOI, form the URL by appending the DOI to http://dx.doi.org/ rather
  than using the URL in your address bar. The DOI for the article in the Brown
  example is 10.1086/660696. If you consulted the article in a library or
  commercial database, you may give the name of the database instead.}
\item Popular magazine article.\autocite[84]{Martin}
\item Newspaper article.\autocite{Niederkorn}
\item Book review.\autocite[16]{Gorman}
\item Thesis or dissertation.\autocite[22--29, 35]{Amundin}
\item Paper presented at a meeting or conference (unpublished).\autocite{Doyle}
\item Letter in a print collection.\autocite[1: 199-200]{Roberts}
\item Document in a print collection.\autocite[6: 19-23]{Leigh}
\item Website/document in an on-line source (unpublished).\autocite{Deane}
\item Kindle or e-reader Book (no page numbers).\autocite[loc. 103]{Earley}
\item Article or chapter in an Edited Book.\autocite[68]{Bulgakov}
\item Reference works (omit from Bibliography).\footnote{\textit{Peloubet's
  Bible Dictionary}, 6th ed., s.v. ``Romans, Epistle to the.''}
\end{itemize}

\section{Special Applications}

\subsection{Examples of Citing the Bible}

Many students struggle with the proper formatting in citing the Bible. When
citing biblical passages, there are some general guidelines to follow that are
important. It is not necessary to write out full citations of verses or
paragraphs from the Bible since your readers can find the references that you
cite. Citations are written in full when the author needs to make a specific
observation, such as when he/she chooses to follow Luke's example in his message
to Theophilus; ``so that you may know the exact truth about the things you have
been taught'' (Luke 1:4).\footnote{Unless otherwise noted, all biblical passages
referenced are in the \textit{English Standard Version} (Wheaton, IL: Crossway,
2008).} You will notice in the footnote below that only a single footnote is
needed when identifying the Bible version, providing the paper cites from only
one translation. All the following biblical references are given in the text of
the paper, not in the footnotes, unless content in the footnote requires
biblical references. If you use multiple translations or versions of the Bible,
then you would have to use one footnote for each new version and use a system of
abbreviations in the text, but only within parentheses (NASB, KJV, NIV, etc.).
The writing is simplified if you choose one version of the Bible and use it
exclusively. Then you can provide a disclaimer footnote to that effect as stated
in footnote number
\hyperref[fn:eight]{\numberstringnum{\getrefnumber{fn:eight}}}.

In the actual text of a paper you need to follow proper grammatical and style
requirements. Here are some correct examples of how to cite references or
allusions from the Bible. Luke wrote to Theophilus in verse four of his first
chapter so that his patron would have a more exact understanding of the details
of the salvation offered also to the Gentiles. Luke claims that he wrote his
Gospel, ``in consecutive order,'' after having ``investigated everything
carefully from the beginning'' (Luke 1:3). In verse one of Chapter One, Luke
seems to be aware of previous Gospel accounts, but in Luke 1:2, he claims that
he has information from eyewitnesses. You will note in the previous example that
you are permitted to use standard biblical references like Luke 1:2 within a
sentence as long as you introduce it as a biblical reference rather than as part
of your text.

The abbreviations for the books of the Bible can be used only in parentheses
within the text or in footnotes.\footnote{Turabian includes a comprehensive list
of abbreviations for the books of the Bible in sections 24.6.1-24.6.4. Some
professors, however, prefer that students consult the SOD's Writing Guide,
Appendix A:
\bluehref{http://www.liberty.edu/media/1162/2015_writingguide/SoD_Writing_Guide_081215.docx}{Sacred
Book Reference List and Capitalization Guide}. Be sure to clarify what your
professor expects.} For example, you may make a reference to Romans 1:16, but if
you state that Christians should not be ashamed of the gospel (Rom 1:16), then
you should use the abbreviation within parentheses. The following examples are
all correct: Paul, in verse sixteen of Chapter One of his Epistle to the Romans,
states that he is not ashamed of the gospel; Paul states that he is not ashamed
of the gospel (Rom 1:16); and in Romans 1:16, Paul states that he is not ashamed
of the gospel.\footnote{Notice the word ``gospel'' is not capitalized when
referring to the evangelical message (i.e. ``good news''). It is capitalized
when referring to one of the first four books of the New Testament.}

\subsection{Study Bible}

Citing quotations from study Bible notes or book introductions is done
differently than citing Bible verses. Citing notes from a study Bible is usually
appropriate only for reflection papers, journal entries, or informal essays. For
exegetical or research papers, commentaries, grammars, and theological works
more specific to the field should be used instead. Ask your professor if it is
appropriate to use Study Bible notes for a particular assignment. Cite study
notes or book introductions in footnotes as you would a chapter in an edited
anthology (Turabian 17.1.8.3). For example:

\begin{singlespace}
\textsuperscript{2}R. Alan Culpepper, ``Luke,'' in \textit{The New Interpreter's
Bible}, Vol. 9, eds. Leander E. Keck, et al. (Nashville, TN: Abingdon Press,
1995), 62-63.
\end{singlespace}

\subsection{Map, Photography, Figure, or Table}

If you are citing a map, photograph, figure, or table, it is also cited in the
footnotes, according to Turabian 17.1.7.2. For example:

\begin{singlespace}
\textsuperscript{2}R. Alan Culpepper, ``Luke,'' in \textit{The New Interpreter's
Bible}, Vol. 9, eds. Leander E. Keck, et al. (Nashville, TN: Abingdon Press,
1995), 89, table 6.4.
\end{singlespace}

\subsection{Crediting Authors of Chapters}

In Turabian format, you are required to credit the author of each individual
chapter that you gleaned material from \dots{} and each of those would be
individual references. See section 19.1.1.1 of your Turabian
manual.\autocite[232]{Turabian}

\subsection{Numbering}

Any number used in the text that is less than one hundred and any whole number
of hundreds should be spelled completely within the body of the paper (one
hundred, two hundred, etc.).\footnote{The exception is within a footnote, where
all numeric numerals can be used (e.g. 100, 200, etc.).} Generally, if the
number can be written with one or two words, it should be spelled completely.
For numbers written with more than two words (i.e., 108 or 210), numerals should
be used. However, you should never mix the styles. If any number used must be
written with numerals, then all should be in the same style (i.e., 98, 108, 210,
300; not ninety-eight, 108, 210, three hundred). Of note here is an exception
that when writing percentages in the text, you would write 98 percent or 100
percent, and so forth; always using the numeral, but writing out ``percent.''

\subsection{Permalinks}

Some resources have permalinks available. They may appear as a paperclip or
linked-chains icon \dots{} or by the word Permalink.
\bluehref{http://umuc.edu/library/libhow/linking.cfm}{This webpage} explains
more. If you cannot find a viable URL that does not require log-on credentials,
then omit the URL altogether and just include all the other elements.

\subsection{Turabian -- Videos}

Videos and podcasts in notes-bibliography style are addressed in section
17.8.3.5 of your Turabian manual.\autocite[204]{Turabian}

\subsection{Turabian -- Ebooks with No Page Numbers}

Section 17.1.10 addresses ebooks. the bottom paragraph on page 181 has all the
specifics.\autocite[181]{Turabian}

\section[eTurabian.com]{Eturabian.com}

Students are encouraged to study and learn the specifics of how to cite each
type of resource. The Online Writing Center has found one online resource to
consistently and dependably produce bibliographic entries:
\bluehref{http://www.eturabian.com/}{www.eturabian.com.} Its software was
generated by an independent entity, but thus far, OWC staff have not discerned
any errors (provided the details are properly input into the citation
generator).

\section{Conclusion}

The conclusion of a paper in Turabian style should reiterate the thesis (though
not necessarily verbatim) and provide the audience with a concise summary of all
the major points. The importance of an effective conclusion cannot be
overstated, as it frames the writer's closing thoughts and should provide a
lasting impression on the reader.

\section{Bibliography Entries and Tips}

The bibliography list itself begins on a new page following the Conclusion even
though a great deal of room may be left on a final page. To do this, hold down
the ``Ctrl'' key and then hit the ``Enter'' key, which will drop the cursor down
to an entirely new page to start the bibliography. Type the word Bibliography,
centered, in bold type, followed by one blank line. The bibliography is
single-spaced but with a blank line (or 12-pt line space after) inserted between
each entry. Chapter 17 of the Turabian manual is dedicated to the various forms
of bibliographic entries.\autocite[164-215]{Turabian}

The following bulleted list depicts a block quote with bullets (single-spaced,
indented \sfrac{1}{2}'') from left margin, with ragged right margin:

\begin{quotation}
\mbox{}\vspace{-\baselineskip}
\begin{itemize}
\item Use the term \textit{Bibliography} for your final list of bibliographic
  entries. Other terms such as \textit{References} or \textit{Works Cited} are
  not acceptable.
\item Bold the title, center it, and begin a new page with normal page
  numbering.
\item Use a one-half inch hanging indentation for the second+ line(s) of each.
\item Only cite sources directly referenced in the body of your paper. Do not
  cite works that have only been consulted. For every reference, there should be
  a footnote and vice-versa.
\item When including two or more works from the same author in the bibliography,
  Turabian \nth{8} edition calls for a long dash, called a 3-em dash. For more
  on this see Turabian, \textit{Manual for Writers}, 151-152.
\item Break the URL at a logical breaking point (after a period, /, etc.) to go
  to the next line. Do this by placing the cursor where you want it to break,
  then click Ctrl-Enter.
\item When consulting an anthology where all the chapters were written by
  different authors, insert the inclusive page numbers of that particular
  chapter in the bibliographic entry. If citing from a book within a collection
  of books, also insert the inclusive page numbers of that particular book in
  the bibliographic entry.
\item When citing an article from an online library/search engine, you do not
  need to cite the search engine or article address if the article is in the
  same form as it would have been in a print journal (typically you can tell
  this is the case if the online article is downloadable into a .pdf). If you
  need to cite the link to the article, it is preferred that you cite the
  article using a Digital Object Identifier (DOI). If no DOI is available, make
  sure you are using a permalink rather than a link copied and pasted from your
  web browser's address bar. If you signed into the Liberty University Jerry
  Falwell Library online using a username and password, then the address from
  the address bar will not work for anyone who does not have \emph{your}
  username/password.
\item Italicize book titles; use quotation marks for article titles.
\item Do not include the Bible in the bibliography. Since the Bible is
  considered a sacred work, cite it initially in the footnotes and subsequently
  in parenthetical references. For example, note the parenthetical reference in
  the following sentence: Christ declares his exclusive salvific value when he
  states, ``I am the way, the truth, and the life'' (John 14:6). The Bible
  reference is not included within the quotation marks because it is not a part
  of Christ's declaration. The period would come after the parenthesis because
  you cannot begin a new sentence without a period immediately preceding it.
\end{itemize}
\end{quotation}

This would be the last page of text in the body of the paper. Even though it
only covers about half the page, drop down to the next page before beginning
your bibliography. Ordinarily, there should be no coloring in any research
paper, and no footnotes in the bibliography list. However, both are included
below to help delineate among the different types of resources, as a teaching
tool for students.

\clearpage
\printbibliography

\end{document}
